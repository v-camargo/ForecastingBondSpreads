% Options for packages loaded elsewhere
\PassOptionsToPackage{unicode}{hyperref}
\PassOptionsToPackage{hyphens}{url}
%
\documentclass[
]{article}
\usepackage{amsmath,amssymb}
\usepackage{iftex}
\ifPDFTeX
  \usepackage[T1]{fontenc}
  \usepackage[utf8]{inputenc}
  \usepackage{textcomp} % provide euro and other symbols
\else % if luatex or xetex
  \usepackage{unicode-math} % this also loads fontspec
  \defaultfontfeatures{Scale=MatchLowercase}
  \defaultfontfeatures[\rmfamily]{Ligatures=TeX,Scale=1}
\fi
\usepackage{lmodern}
\ifPDFTeX\else
  % xetex/luatex font selection
\fi
% Use upquote if available, for straight quotes in verbatim environments
\IfFileExists{upquote.sty}{\usepackage{upquote}}{}
\IfFileExists{microtype.sty}{% use microtype if available
  \usepackage[]{microtype}
  \UseMicrotypeSet[protrusion]{basicmath} % disable protrusion for tt fonts
}{}
\makeatletter
\@ifundefined{KOMAClassName}{% if non-KOMA class
  \IfFileExists{parskip.sty}{%
    \usepackage{parskip}
  }{% else
    \setlength{\parindent}{0pt}
    \setlength{\parskip}{6pt plus 2pt minus 1pt}}
}{% if KOMA class
  \KOMAoptions{parskip=half}}
\makeatother
\usepackage{xcolor}
\usepackage[margin=1in]{geometry}
\usepackage{color}
\usepackage{fancyvrb}
\newcommand{\VerbBar}{|}
\newcommand{\VERB}{\Verb[commandchars=\\\{\}]}
\DefineVerbatimEnvironment{Highlighting}{Verbatim}{commandchars=\\\{\}}
% Add ',fontsize=\small' for more characters per line
\usepackage{framed}
\definecolor{shadecolor}{RGB}{248,248,248}
\newenvironment{Shaded}{\begin{snugshade}}{\end{snugshade}}
\newcommand{\AlertTok}[1]{\textcolor[rgb]{0.94,0.16,0.16}{#1}}
\newcommand{\AnnotationTok}[1]{\textcolor[rgb]{0.56,0.35,0.01}{\textbf{\textit{#1}}}}
\newcommand{\AttributeTok}[1]{\textcolor[rgb]{0.13,0.29,0.53}{#1}}
\newcommand{\BaseNTok}[1]{\textcolor[rgb]{0.00,0.00,0.81}{#1}}
\newcommand{\BuiltInTok}[1]{#1}
\newcommand{\CharTok}[1]{\textcolor[rgb]{0.31,0.60,0.02}{#1}}
\newcommand{\CommentTok}[1]{\textcolor[rgb]{0.56,0.35,0.01}{\textit{#1}}}
\newcommand{\CommentVarTok}[1]{\textcolor[rgb]{0.56,0.35,0.01}{\textbf{\textit{#1}}}}
\newcommand{\ConstantTok}[1]{\textcolor[rgb]{0.56,0.35,0.01}{#1}}
\newcommand{\ControlFlowTok}[1]{\textcolor[rgb]{0.13,0.29,0.53}{\textbf{#1}}}
\newcommand{\DataTypeTok}[1]{\textcolor[rgb]{0.13,0.29,0.53}{#1}}
\newcommand{\DecValTok}[1]{\textcolor[rgb]{0.00,0.00,0.81}{#1}}
\newcommand{\DocumentationTok}[1]{\textcolor[rgb]{0.56,0.35,0.01}{\textbf{\textit{#1}}}}
\newcommand{\ErrorTok}[1]{\textcolor[rgb]{0.64,0.00,0.00}{\textbf{#1}}}
\newcommand{\ExtensionTok}[1]{#1}
\newcommand{\FloatTok}[1]{\textcolor[rgb]{0.00,0.00,0.81}{#1}}
\newcommand{\FunctionTok}[1]{\textcolor[rgb]{0.13,0.29,0.53}{\textbf{#1}}}
\newcommand{\ImportTok}[1]{#1}
\newcommand{\InformationTok}[1]{\textcolor[rgb]{0.56,0.35,0.01}{\textbf{\textit{#1}}}}
\newcommand{\KeywordTok}[1]{\textcolor[rgb]{0.13,0.29,0.53}{\textbf{#1}}}
\newcommand{\NormalTok}[1]{#1}
\newcommand{\OperatorTok}[1]{\textcolor[rgb]{0.81,0.36,0.00}{\textbf{#1}}}
\newcommand{\OtherTok}[1]{\textcolor[rgb]{0.56,0.35,0.01}{#1}}
\newcommand{\PreprocessorTok}[1]{\textcolor[rgb]{0.56,0.35,0.01}{\textit{#1}}}
\newcommand{\RegionMarkerTok}[1]{#1}
\newcommand{\SpecialCharTok}[1]{\textcolor[rgb]{0.81,0.36,0.00}{\textbf{#1}}}
\newcommand{\SpecialStringTok}[1]{\textcolor[rgb]{0.31,0.60,0.02}{#1}}
\newcommand{\StringTok}[1]{\textcolor[rgb]{0.31,0.60,0.02}{#1}}
\newcommand{\VariableTok}[1]{\textcolor[rgb]{0.00,0.00,0.00}{#1}}
\newcommand{\VerbatimStringTok}[1]{\textcolor[rgb]{0.31,0.60,0.02}{#1}}
\newcommand{\WarningTok}[1]{\textcolor[rgb]{0.56,0.35,0.01}{\textbf{\textit{#1}}}}
\usepackage{graphicx}
\makeatletter
\def\maxwidth{\ifdim\Gin@nat@width>\linewidth\linewidth\else\Gin@nat@width\fi}
\def\maxheight{\ifdim\Gin@nat@height>\textheight\textheight\else\Gin@nat@height\fi}
\makeatother
% Scale images if necessary, so that they will not overflow the page
% margins by default, and it is still possible to overwrite the defaults
% using explicit options in \includegraphics[width, height, ...]{}
\setkeys{Gin}{width=\maxwidth,height=\maxheight,keepaspectratio}
% Set default figure placement to htbp
\makeatletter
\def\fps@figure{htbp}
\makeatother
\setlength{\emergencystretch}{3em} % prevent overfull lines
\providecommand{\tightlist}{%
  \setlength{\itemsep}{0pt}\setlength{\parskip}{0pt}}
\setcounter{secnumdepth}{-\maxdimen} % remove section numbering
\ifLuaTeX
  \usepackage{selnolig}  % disable illegal ligatures
\fi
\IfFileExists{bookmark.sty}{\usepackage{bookmark}}{\usepackage{hyperref}}
\IfFileExists{xurl.sty}{\usepackage{xurl}}{} % add URL line breaks if available
\urlstyle{same}
\hypersetup{
  pdftitle={ForecastBonds},
  pdfauthor={Risco - Augme},
  hidelinks,
  pdfcreator={LaTeX via pandoc}}

\title{ForecastBonds}
\author{Risco - Augme}
\date{2023-07-07}

\begin{document}
\maketitle

\hypertarget{bibliotecas-utilizadas}{%
\subsection{Bibliotecas Utilizadas}\label{bibliotecas-utilizadas}}

\begin{Shaded}
\begin{Highlighting}[]
\FunctionTok{library}\NormalTok{(tidyverse)}
\FunctionTok{library}\NormalTok{(rugarch)}
\FunctionTok{library}\NormalTok{(fGarch)}
\FunctionTok{library}\NormalTok{(timeSeries)}
\FunctionTok{library}\NormalTok{(fpp3)}
\FunctionTok{library}\NormalTok{(gridExtra)}
\FunctionTok{library}\NormalTok{(purrr)}
\FunctionTok{library}\NormalTok{(bizdays)}
\end{Highlighting}
\end{Shaded}

\hypertarget{definiuxe7uxe3o-de-funuxe7uxf5es}{%
\subsection{Definição de
Funções}\label{definiuxe7uxe3o-de-funuxe7uxf5es}}

Utilizaremos estas funções para ajustar o modelo através do grid search:

\begin{Shaded}
\begin{Highlighting}[]
\DocumentationTok{\#\# Função para fitting do modelo GARCH especificado}
\NormalTok{GModels }\OtherTok{\textless{}{-}} \ControlFlowTok{function}\NormalTok{(parms, series, }\AttributeTok{prog =} \ConstantTok{NULL}\NormalTok{)\{}
  
  \ControlFlowTok{if}\NormalTok{ (}\SpecialCharTok{!}\FunctionTok{is.null}\NormalTok{(prog)) }\FunctionTok{prog}\NormalTok{()}
  
  \CommentTok{\#configurando o modelo iGARCH}
  \ControlFlowTok{if}\NormalTok{(parms}\SpecialCharTok{$}\NormalTok{model}\SpecialCharTok{==}\StringTok{"iGARCH"}\NormalTok{)\{}
\NormalTok{    garch\_model }\OtherTok{=} \FunctionTok{ugarchspec}\NormalTok{(}
      \AttributeTok{variance.model =} \FunctionTok{list}\NormalTok{(}\AttributeTok{model=}\NormalTok{parms}\SpecialCharTok{$}\NormalTok{model, }\AttributeTok{garchOrder=}\FunctionTok{c}\NormalTok{(parms}\SpecialCharTok{$}\NormalTok{m, parms}\SpecialCharTok{$}\NormalTok{n)),}
      \AttributeTok{mean.model =} \FunctionTok{list}\NormalTok{(}\AttributeTok{armaOrder =} \FunctionTok{c}\NormalTok{(parms}\SpecialCharTok{$}\NormalTok{p, parms}\SpecialCharTok{$}\NormalTok{q), }\AttributeTok{include.mean =} \ConstantTok{TRUE}\NormalTok{),}
      \AttributeTok{distribution.model =}\NormalTok{ parms}\SpecialCharTok{$}\NormalTok{dist)}
\NormalTok{  \}}
  
  \CommentTok{\#configurando os outros modelos}
  \ControlFlowTok{else}\NormalTok{\{}
\NormalTok{    garch\_model }\OtherTok{=} \FunctionTok{ugarchspec}\NormalTok{(}
      \AttributeTok{variance.model =} \FunctionTok{list}\NormalTok{(}\AttributeTok{model=}\NormalTok{parms}\SpecialCharTok{$}\NormalTok{model, }\AttributeTok{submodel=}\NormalTok{parms}\SpecialCharTok{$}\NormalTok{submodel, }\AttributeTok{garchOrder=}\FunctionTok{c}\NormalTok{(parms}\SpecialCharTok{$}\NormalTok{m, parms}\SpecialCharTok{$}\NormalTok{n)),}
      \AttributeTok{mean.model =} \FunctionTok{list}\NormalTok{(}\AttributeTok{armaOrder =} \FunctionTok{c}\NormalTok{(parms}\SpecialCharTok{$}\NormalTok{p, parms}\SpecialCharTok{$}\NormalTok{q), }\AttributeTok{include.mean =} \ConstantTok{TRUE}\NormalTok{),}
      \AttributeTok{distribution.model =}\NormalTok{ parms}\SpecialCharTok{$}\NormalTok{dist}
\NormalTok{    )}
\NormalTok{  \}}
  
  \CommentTok{\# ocultando os avisos quando o modelo não convergir}
  \FunctionTok{suppressWarnings}\NormalTok{(\{fit }\OtherTok{\textless{}{-}} \FunctionTok{ugarchfit}\NormalTok{(}\AttributeTok{spec=}\NormalTok{garch\_model, }\AttributeTok{data=}\NormalTok{series, }\AttributeTok{solver=}\StringTok{\textquotesingle{}solnp\textquotesingle{}}\NormalTok{, }\AttributeTok{solver.control=}\FunctionTok{list}\NormalTok{(}\AttributeTok{tol =} \FloatTok{5e{-}8}\NormalTok{))\})}
  
  
  \CommentTok{\# fitting do modelo}
\NormalTok{  fit}
\NormalTok{\}}


\DocumentationTok{\#\# Função para encontrar o melhor os melhores parametros para a família de moodelos GARCH:}
\NormalTok{find\_best\_garch }\OtherTok{\textless{}{-}} \ControlFlowTok{function}\NormalTok{(ativo, grid, df)\{}
  \CommentTok{\#seleciona o retorno do ativo}
\NormalTok{  retDiario }\OtherTok{\textless{}{-}}\NormalTok{ bondsTibble }\SpecialCharTok{|\textgreater{}}\NormalTok{ dplyr}\SpecialCharTok{::}\FunctionTok{filter}\NormalTok{(Ativo}\SpecialCharTok{==}\NormalTok{ativo) }\SpecialCharTok{|\textgreater{}}\NormalTok{ dplyr}\SpecialCharTok{::}\FunctionTok{select}\NormalTok{(rDiario) }\SpecialCharTok{|\textgreater{}} \FunctionTok{pull}\NormalTok{()}
  
  \CommentTok{\#deixa visível ao usuário o progresso do ajuste do modelo}
\NormalTok{  usethis}\SpecialCharTok{::}\FunctionTok{ui\_info}\NormalTok{(}\StringTok{"Adjusting models for \{ativo\}..."}\NormalTok{)}

\NormalTok{  progressr}\SpecialCharTok{::}\FunctionTok{with\_progress}\NormalTok{(\{}
\NormalTok{    prog }\OtherTok{\textless{}{-}}\NormalTok{ progressr}\SpecialCharTok{::}\FunctionTok{progressor}\NormalTok{(}\FunctionTok{nrow}\NormalTok{(grid))}
\NormalTok{    models }\OtherTok{\textless{}{-}}\NormalTok{ grid }\SpecialCharTok{\%\textgreater{}\%}
      \FunctionTok{group\_split}\NormalTok{(id) }\SpecialCharTok{\%\textgreater{}\%}
\NormalTok{      purrr}\SpecialCharTok{::}\FunctionTok{map}\NormalTok{(GModels, }\AttributeTok{series=}\NormalTok{retDiario, prog)}
\NormalTok{  \})}
\NormalTok{  safe\_info }\OtherTok{\textless{}{-}}\NormalTok{ purrr}\SpecialCharTok{::}\FunctionTok{possibly}\NormalTok{(infocriteria, tibble}\SpecialCharTok{::}\FunctionTok{tibble}\NormalTok{())}
  
  \CommentTok{\#agrega as informações de parâmetros dos modelos testados}
  \FunctionTok{suppressWarnings}\NormalTok{(\{}
\NormalTok{    info }\OtherTok{\textless{}{-}}\NormalTok{ purrr}\SpecialCharTok{::}\FunctionTok{map}\NormalTok{(models, safe\_info) }\SpecialCharTok{\%\textgreater{}\%}
\NormalTok{      purrr}\SpecialCharTok{::}\FunctionTok{map}\NormalTok{(tibble}\SpecialCharTok{::}\NormalTok{as\_tibble, }\AttributeTok{rownames =} \StringTok{"criteria"}\NormalTok{) }\SpecialCharTok{\%\textgreater{}\%}
\NormalTok{      dplyr}\SpecialCharTok{::}\FunctionTok{bind\_rows}\NormalTok{(}\AttributeTok{.id =} \StringTok{"id"}\NormalTok{)}
\NormalTok{  \})}
  \FunctionTok{return}\NormalTok{(info)}
  
  \CommentTok{\#seleciona os parâmetros do melhor modelo através do critério de Akaike}
\NormalTok{  best }\OtherTok{\textless{}{-}}\NormalTok{ info }\SpecialCharTok{|\textgreater{}} 
    \FunctionTok{inner\_join}\NormalTok{(grid, }\StringTok{"id"}\NormalTok{) }\SpecialCharTok{|\textgreater{}} 
    \FunctionTok{pivot\_wider}\NormalTok{(}\AttributeTok{names\_from =}\NormalTok{ criteria, }\AttributeTok{values\_from =}\NormalTok{ V1) }\SpecialCharTok{|\textgreater{}} 
\NormalTok{    janitor}\SpecialCharTok{::}\FunctionTok{clean\_names}\NormalTok{() }\SpecialCharTok{|\textgreater{}} 
    \FunctionTok{arrange}\NormalTok{(akaike)}
  
  
  \CommentTok{\#seleciona os melhores paramestros}
\NormalTok{  usethis}\SpecialCharTok{::}\FunctionTok{ui\_info}\NormalTok{(}\FunctionTok{c}\NormalTok{(}
    \StringTok{"Best model:"}\NormalTok{,}
    \StringTok{"p \textless{}{-} \{best$p[1]\}"}\NormalTok{,}
    \StringTok{"q \textless{}{-} \{best$q[1]\}"}\NormalTok{,}
    \StringTok{"m \textless{}{-} \{best$m[1]\}"}\NormalTok{,}
    \StringTok{"n \textless{}{-} \{best$n[1]\}"}
\NormalTok{  ))}
  
\NormalTok{  best}
  
\NormalTok{\}}
\end{Highlighting}
\end{Shaded}

\hypertarget{importauxe7uxe3o-e-transformauxe7uxe3o-dos-dados}{%
\subsection{Importação e Transformação dos
Dados}\label{importauxe7uxe3o-e-transformauxe7uxe3o-dos-dados}}

Os dados devem ser importados via planilha \textbf{importBonds} em excel

\begin{Shaded}
\begin{Highlighting}[]
\NormalTok{rawDataFinal }\OtherTok{\textless{}{-}}\NormalTok{ readxl}\SpecialCharTok{::}\FunctionTok{read\_xlsx}\NormalTok{(}\StringTok{"importBonds.xlsx"}\NormalTok{, }
                                   \AttributeTok{sheet =} \StringTok{"import"}\NormalTok{,}
                                   \AttributeTok{skip =} \DecValTok{3}\NormalTok{)}
\CommentTok{\#transforma a tabela em formato tidy data}
\NormalTok{tidyData }\OtherTok{\textless{}{-}}\NormalTok{ rawDataFinal }\SpecialCharTok{|\textgreater{}} 
            \FunctionTok{mutate\_if}\NormalTok{(is.double, as.character) }\SpecialCharTok{|\textgreater{}} 
            \FunctionTok{pivot\_longer}\NormalTok{(}\AttributeTok{cols =} \SpecialCharTok{!}\NormalTok{Dates, }\AttributeTok{names\_to =} \StringTok{"Ativo"}\NormalTok{)}

\CommentTok{\#retira N/A e converte as colunas para o tipo correto }
\NormalTok{tidyData }\OtherTok{\textless{}{-}}\NormalTok{ tidyData }\SpecialCharTok{|\textgreater{}}\NormalTok{ dplyr}\SpecialCharTok{::}\FunctionTok{filter}\NormalTok{(}\SpecialCharTok{!}\FunctionTok{grepl}\NormalTok{(}\StringTok{"N/A"}\NormalTok{, value))}
\NormalTok{tidyData }\OtherTok{\textless{}{-}}\NormalTok{ tidyData }\SpecialCharTok{|\textgreater{}} \FunctionTok{mutate}\NormalTok{(}\AttributeTok{Dates=}\FunctionTok{as.Date}\NormalTok{(Dates),}
                               \AttributeTok{value =} \FunctionTok{as.double}\NormalTok{(value))}

\CommentTok{\#cria o df de time series}
\NormalTok{bondsTS }\OtherTok{\textless{}{-}}\NormalTok{ tidyData }\SpecialCharTok{|\textgreater{}} \FunctionTok{as\_tsibble}\NormalTok{(}\AttributeTok{key =}\NormalTok{ Ativo, }\AttributeTok{index =}\NormalTok{ Dates, }\AttributeTok{regular =} \ConstantTok{FALSE}\NormalTok{)}

\CommentTok{\#cria variável para controlar os ativos que estão sendo analisados}
\NormalTok{ativosAnalisados }\OtherTok{\textless{}{-}}\NormalTok{ tidyData}\SpecialCharTok{$}\NormalTok{Ativo }\SpecialCharTok{|\textgreater{}} \FunctionTok{unique}\NormalTok{()}

\CommentTok{\#Adicionando o retorno dos ativos}
\NormalTok{bondsTS }\OtherTok{\textless{}{-}}\NormalTok{ bondsTS }\SpecialCharTok{|\textgreater{}}\FunctionTok{group\_by}\NormalTok{(Ativo) }\SpecialCharTok{|\textgreater{}} \FunctionTok{mutate}\NormalTok{(}\AttributeTok{rDiario =}\NormalTok{ dplyr}\SpecialCharTok{::}\FunctionTok{lag}\NormalTok{(value)}\SpecialCharTok{/}\NormalTok{value}\DecValTok{{-}1}\NormalTok{)}

\CommentTok{\#retirando o primeiro dia (retorno = NA)}
\NormalTok{bondsTSClean }\OtherTok{\textless{}{-}}\NormalTok{ bondsTS }\SpecialCharTok{|\textgreater{}} \FunctionTok{group\_by}\NormalTok{(Ativo) }\SpecialCharTok{|\textgreater{}} 
\NormalTok{                dplyr}\SpecialCharTok{::}\FunctionTok{filter}\NormalTok{(}\FunctionTok{is.na}\NormalTok{(rDiario) }\SpecialCharTok{==} \ConstantTok{FALSE}\NormalTok{)}
\end{Highlighting}
\end{Shaded}

\hypertarget{suxe9ries-analisadas}{%
\subsection{Séries Analisadas}\label{suxe9ries-analisadas}}

\begin{Shaded}
\begin{Highlighting}[]
\NormalTok{bondsTSClean }\SpecialCharTok{|\textgreater{}} \FunctionTok{autoplot}\NormalTok{(value)}
\end{Highlighting}
\end{Shaded}

\includegraphics{ForecastBonds_files/figure-latex/series plot-1.pdf}

\begin{Shaded}
\begin{Highlighting}[]
\NormalTok{bondsTSClean }\SpecialCharTok{|\textgreater{}} \FunctionTok{autoplot}\NormalTok{(rDiario) }\SpecialCharTok{+}
          \FunctionTok{facet\_wrap}\NormalTok{(}\SpecialCharTok{\textasciitilde{}}\NormalTok{Ativo,}\AttributeTok{ncol =} \DecValTok{2}\NormalTok{)}
\end{Highlighting}
\end{Shaded}

\includegraphics{ForecastBonds_files/figure-latex/series plot-2.pdf}

\hypertarget{gruxe1ficos-de-autocorrelauxe7uxe3o-dos-retornos}{%
\subsubsection{Gráficos de Autocorrelação dos
Retornos}\label{gruxe1ficos-de-autocorrelauxe7uxe3o-dos-retornos}}

No geral, a autocorrelação de uma série decai para zero quanto maior a
defasagem, mas, no caso de um processo estacionário, a autocorrelação
decai muito rápido para zero conforme a defasagem aumenta (geralmente de
maneira exponencial ou com senoides amortecidas). Precisamos entender se
as séries com que estamos trabalhando apresentam comportamento de ruído
branco ou não

\begin{Shaded}
\begin{Highlighting}[]
\CommentTok{\#ACF}
\NormalTok{bondsTSClean  }\SpecialCharTok{|\textgreater{}}  \FunctionTok{ACF}\NormalTok{(rDiario) }\SpecialCharTok{|\textgreater{}} \FunctionTok{autoplot}\NormalTok{()}
\end{Highlighting}
\end{Shaded}

\includegraphics{ForecastBonds_files/figure-latex/ACF-1.pdf}

\begin{Shaded}
\begin{Highlighting}[]
\CommentTok{\#PACF}
\NormalTok{bondsTSClean }\SpecialCharTok{|\textgreater{}} \FunctionTok{PACF}\NormalTok{(rDiario) }\SpecialCharTok{|\textgreater{}} \FunctionTok{autoplot}\NormalTok{()}
\end{Highlighting}
\end{Shaded}

\includegraphics{ForecastBonds_files/figure-latex/ACF-2.pdf} Em
finanças, como fato estilizado, os retornos dos ativos se comportam de
maneira parecida com um ruído branco, sendo assim, analisar somente os
gráficos ACF e PACF pode nos levar a conclusões errôneas. Para testarmos
se a série com a qual trabalharemos nos modelos é um ruído branco ou
não, podemos utilizar o teste estatístico de Ljung-Box.

\hypertarget{teste-de-ljung-box}{%
\subsubsection{Teste de Ljung-Box}\label{teste-de-ljung-box}}

Este teste visa entender se a série estudada se comporta como ruído
branco através do seguinte teste de hipótese:

\begin{itemize}
\tightlist
\item
  H0: a série \textbf{se comporta como um ruído branco}
\item
  H1: a série \textbf{não se comporta como um ruído branco}
\end{itemize}

Em outras palavras, caso o p-valor seja substancialmente pequeno
(\textless0.05), podemos concluir que a série não se comporta como um
ruído branco, sugerindo que a série tem autocorrelação.

\begin{Shaded}
\begin{Highlighting}[]
\CommentTok{\#criando o df que será utilizado para o teste e para o posterior modelo}
\NormalTok{bondsTibble }\OtherTok{\textless{}{-}}\NormalTok{ bondsTSClean }\SpecialCharTok{|\textgreater{}} \FunctionTok{as\_tibble}\NormalTok{()}

\CommentTok{\#realizando o teste de LJung{-}Box}
\NormalTok{whiteNoiseTest }\OtherTok{\textless{}{-}} \FunctionTok{tibble}\NormalTok{(}\AttributeTok{Ativo=}\FunctionTok{as.character}\NormalTok{(), }\AttributeTok{pValor=}\FunctionTok{as.double}\NormalTok{())}
\ControlFlowTok{for}\NormalTok{ (j }\ControlFlowTok{in}\NormalTok{ ativosAnalisados)\{}
\NormalTok{  rAtivo }\OtherTok{\textless{}{-}}\NormalTok{ bondsTibble }\SpecialCharTok{|\textgreater{}}\NormalTok{ dplyr}\SpecialCharTok{::}\FunctionTok{filter}\NormalTok{(Ativo }\SpecialCharTok{==}\NormalTok{ j) }\SpecialCharTok{|\textgreater{}}\NormalTok{ dplyr}\SpecialCharTok{::}\FunctionTok{select}\NormalTok{(rDiario)}
\NormalTok{  Ljung }\OtherTok{\textless{}{-}}\FunctionTok{Box.test}\NormalTok{(rAtivo, }\AttributeTok{lag=}\DecValTok{12}\NormalTok{, }\AttributeTok{fitdf=}\DecValTok{1}\NormalTok{, }\AttributeTok{type=}\StringTok{"Ljung{-}Box"}\NormalTok{)}
\NormalTok{  whiteNoiseTest }\OtherTok{\textless{}{-}}\NormalTok{ whiteNoiseTest }\SpecialCharTok{|\textgreater{}}\NormalTok{ dplyr}\SpecialCharTok{::}\FunctionTok{bind\_rows}\NormalTok{(}\FunctionTok{tibble}\NormalTok{(}\AttributeTok{Ativo=}\NormalTok{j,}\AttributeTok{pValor=}\NormalTok{Ljung}\SpecialCharTok{$}\NormalTok{p.value)) }\SpecialCharTok{|\textgreater{}} \FunctionTok{as.data.frame}\NormalTok{()}
\NormalTok{\}}

\NormalTok{whiteNoiseTest}
\end{Highlighting}
\end{Shaded}

\begin{verbatim}
##      Ativo       pValor
## 1 ITAUBZ17 0.000000e+00
## 2 CMIGBZ24 0.000000e+00
## 3 ITAUBZ18 0.000000e+00
## 4 SIMPAR31 3.050515e-11
## 5 MOVIBZ31 1.394549e-10
## 6 HIDRVS31 0.000000e+00
## 7 PRIOBZ26 0.000000e+00
## 8   STNE28 1.910916e-12
\end{verbatim}

\hypertarget{modelos-de-heterocedasticidade-condicional-auto-regressiva-generalizada-garch}{%
\subsection{Modelos de Heterocedasticidade Condicional Auto-Regressiva
Generalizada
(GARCH)}\label{modelos-de-heterocedasticidade-condicional-auto-regressiva-generalizada-garch}}

\hypertarget{grid-search---definindo-o-melhor-modelo-para-cada-suxe9rie}{%
\subsubsection{Grid Search - Definindo o melhor modelo para cada
série}\label{grid-search---definindo-o-melhor-modelo-para-cada-suxe9rie}}

Esta família de modelos tratam séries com variância não constante ao
longo do tempo (heterocedasticidade). Para modelarmos a série, iremos
utilizar a distribuição \textbf{t-student} por ser simétrica com caudas
mais pesadas (distribuição leptocúrtica), característica observada nos
retornos dos ativos do mercado financeiro.

Iremos testar diferentes modelos e parâmetros para escolher o melhor
através do menor critério de Akaike. Será realizada uma combinação do
modelo fGARCH com os submodelos TGARCH e GARCH com os valores de
\textbf{m} e \textbf{n} variando entre 0 e 2 e com os valores de
\textbf{p} e \textbf{q} variando entre 0 e 3.

\begin{Shaded}
\begin{Highlighting}[]
\CommentTok{\#Para testar os diferentes modelos de volatilidade, iremos utilizar a seguinte tabela}
\CommentTok{\#mtr \textless{}{-} crossing(m=c(0:2), n=c(0:2), p=c(0:3), q=c(0:3), dist=c("std"))}
\CommentTok{\#GridSearch \textless{}{-} bind\_rows(cbind(tibble(model="fGARCH", submodel="TGARCH"), mtr),}
\CommentTok{\#                        cbind(tibble(model="fGARCH", submodel="GARCH"), mtr)) \%\textgreater{}\%}
\CommentTok{\#  tibble::rownames\_to\_column("id")}

\CommentTok{\#Rodando cada linha especificação do GridSearch nos modelos}
\CommentTok{\#bestModels \textless{}{-} ativosAnalisados |\textgreater{} }
\CommentTok{\#  set\_names() |\textgreater{} }
\CommentTok{\#  map(find\_best\_garch, grid=GridSearch) |\textgreater{} }
\CommentTok{\#  bind\_rows(.id="Ativo")}

\CommentTok{\#Escolhando a melhor especificação para cada ativo através do Critério de Akaike}
\CommentTok{\#modeloFinal \textless{}{-}  bestModels \%\textgreater{}\%}
\CommentTok{\#  dplyr::filter(criteria == \textquotesingle{}Akaike\textquotesingle{}) |\textgreater{} }
\CommentTok{\#  group\_by(Ativo) |\textgreater{} }
\CommentTok{\#  slice(which.min(V1)) |\textgreater{} }
\CommentTok{\#  merge(GridSearch, by.x="id", by.y="id", all.x=FALSE, all.y=FALSE) |\textgreater{} }
\CommentTok{\#  select({-}id) |\textgreater{} }
\CommentTok{\#  rename(criteriaValue=V1)}
  

\NormalTok{modeloFinal }\OtherTok{\textless{}{-}} \FunctionTok{tibble}\NormalTok{(}\AttributeTok{Ativo=}\FunctionTok{c}\NormalTok{(}\StringTok{"ITAUBZ18"}\NormalTok{, }\StringTok{"PRIOBZ26"}\NormalTok{, }\StringTok{"ITAUBZ17"}\NormalTok{,}\StringTok{"STNE28"}\NormalTok{,}\StringTok{"MOVIBZ31"}\NormalTok{,}\StringTok{"SIMPAR31"}\NormalTok{, }\StringTok{"CMIGBZ24"}\NormalTok{,}\StringTok{"HIDRVS31"}\NormalTok{),}
             \AttributeTok{criteria =} \StringTok{"Akaike"}\NormalTok{,}
             \AttributeTok{criteriaValue =}
               \FunctionTok{c}\NormalTok{(}\SpecialCharTok{{-}}\FloatTok{10.031020}\NormalTok{,}\SpecialCharTok{{-}}\FloatTok{9.597698}\NormalTok{,}\SpecialCharTok{{-}}\FloatTok{9.773093}\NormalTok{,}\SpecialCharTok{{-}}\FloatTok{8.156528}\NormalTok{,}\SpecialCharTok{{-}}\FloatTok{8.245174}\NormalTok{,}\SpecialCharTok{{-}}\FloatTok{8.243398}\NormalTok{,}\SpecialCharTok{{-}}\FloatTok{10.194153}\NormalTok{, }\SpecialCharTok{{-}}\FloatTok{9.042531}\NormalTok{),}
             \AttributeTok{model=}\StringTok{"fGARCH"}\NormalTok{,}
             \AttributeTok{submodel=}\FunctionTok{c}\NormalTok{(}\StringTok{"GARCH"}\NormalTok{,}\StringTok{"GARCH"}\NormalTok{,}\StringTok{"GARCH"}\NormalTok{,}\StringTok{"GARCH"}\NormalTok{,}\StringTok{"GARCH"}\NormalTok{,}\StringTok{"GARCH"}\NormalTok{,}\StringTok{"TGARCH"}\NormalTok{,}\StringTok{"TGARCH"}\NormalTok{),}
             \AttributeTok{m=}\FunctionTok{c}\NormalTok{(}\DecValTok{2}\NormalTok{,}\DecValTok{2}\NormalTok{,}\DecValTok{2}\NormalTok{,}\DecValTok{2}\NormalTok{,}\DecValTok{2}\NormalTok{,}\DecValTok{2}\NormalTok{,}\DecValTok{0}\NormalTok{,}\DecValTok{0}\NormalTok{),}
             \AttributeTok{n=}\FunctionTok{c}\NormalTok{(}\DecValTok{1}\NormalTok{,}\DecValTok{1}\NormalTok{,}\DecValTok{2}\NormalTok{,}\DecValTok{2}\NormalTok{,}\DecValTok{2}\NormalTok{,}\DecValTok{2}\NormalTok{,}\DecValTok{1}\NormalTok{,}\DecValTok{2}\NormalTok{),}
             \AttributeTok{p=}\FunctionTok{c}\NormalTok{(}\DecValTok{0}\NormalTok{,}\DecValTok{2}\NormalTok{,}\DecValTok{0}\NormalTok{,}\DecValTok{1}\NormalTok{,}\DecValTok{2}\NormalTok{,}\DecValTok{3}\NormalTok{,}\DecValTok{3}\NormalTok{,}\DecValTok{0}\NormalTok{),}
             \AttributeTok{q=}\FunctionTok{c}\NormalTok{(}\DecValTok{0}\NormalTok{,}\DecValTok{1}\NormalTok{,}\DecValTok{2}\NormalTok{,}\DecValTok{2}\NormalTok{,}\DecValTok{0}\NormalTok{,}\DecValTok{0}\NormalTok{,}\DecValTok{3}\NormalTok{,}\DecValTok{0}\NormalTok{),}
             \AttributeTok{dist=}\StringTok{"std"}
             
\NormalTok{             )}

\NormalTok{modeloFinal}
\end{Highlighting}
\end{Shaded}

\begin{verbatim}
## # A tibble: 8 x 10
##   Ativo    criteria criteriaValue model  submodel     m     n     p     q dist 
##   <chr>    <chr>            <dbl> <chr>  <chr>    <dbl> <dbl> <dbl> <dbl> <chr>
## 1 ITAUBZ18 Akaike          -10.0  fGARCH GARCH        2     1     0     0 std  
## 2 PRIOBZ26 Akaike           -9.60 fGARCH GARCH        2     1     2     1 std  
## 3 ITAUBZ17 Akaike           -9.77 fGARCH GARCH        2     2     0     2 std  
## 4 STNE28   Akaike           -8.16 fGARCH GARCH        2     2     1     2 std  
## 5 MOVIBZ31 Akaike           -8.25 fGARCH GARCH        2     2     2     0 std  
## 6 SIMPAR31 Akaike           -8.24 fGARCH GARCH        2     2     3     0 std  
## 7 CMIGBZ24 Akaike          -10.2  fGARCH TGARCH       0     1     3     3 std  
## 8 HIDRVS31 Akaike           -9.04 fGARCH TGARCH       0     2     0     0 std
\end{verbatim}

\hypertarget{simulando---prediuxe7uxe3o-dos-retornos-futuros}{%
\subsubsection{Simulando - Predição dos retornos
futuros}\label{simulando---prediuxe7uxe3o-dos-retornos-futuros}}

Tendo as especificações finais de cada modelo para cada série de retorno
analisada, iremos simular \textbf{100 cenários} futuros para os próximos
\textbf{60 dias} e agregar os resultados pela média das predições das
simulações em cada data futura.

\begin{Shaded}
\begin{Highlighting}[]
\CommentTok{\#df para salvar os resultados}
\CommentTok{\#saveResult \textless{}{-} tibble(dataFutura=as.character(),}
\CommentTok{\#                     Ativo=as.character(), }
\CommentTok{\#                     rPrevisto=as.double())}

\CommentTok{\#for (i in seq(1,nrow(modeloFinal)))\{}

\CommentTok{\#o modelo roda somente no df com o retorno do ativo:}
\CommentTok{\#rAtivo \textless{}{-} bondsTibble |\textgreater{}}
\CommentTok{\#  dplyr::filter(Ativo == modeloFinal[i,]$Ativo) |\textgreater{}}
\CommentTok{\#  select(Dates, rDiario) |\textgreater{}}
\CommentTok{\#  column\_to\_rownames(var = \textquotesingle{}Dates\textquotesingle{})}

\CommentTok{\#especificação do modelo conforme o grid search: }
\CommentTok{\#specModel \textless{}{-} ugarchspec(}
\CommentTok{\#              variance.model = list(model=modeloFinal[i,]$model, \#submodel=modeloFinal[i,]$submodel, garchOrder=c(modeloFinal[i,]$m, \#modeloFinal[i,]$n)),}
\CommentTok{\#              mean.model = list(armaOrder = c(modeloFinal[i,]$p, modeloFinal[i,]$q), \#include.mean = TRUE),}
\CommentTok{\#              distribution.model = modeloFinal[i,]$dist)}

\CommentTok{\#fitando o modelo especificado }
\CommentTok{\#fitModel \textless{}{-} ugarchfit(}
\CommentTok{\#            spec = specModel,}
\CommentTok{\#            data = rAtivo,}
\CommentTok{\#            solver = \textquotesingle{}hybrid\textquotesingle{})}
            \CommentTok{\#solver.control = list(tol = 5e{-}8))}
            \CommentTok{\#out.sample = 10)}

\CommentTok{\#simulando diversos cenários }
\CommentTok{\#simuModel \textless{}{-} ugarchsim(fitModel,}
\CommentTok{\#                       n.sim = 60,}
\CommentTok{\#                       m.sim = 100,}
\CommentTok{\#                       rseed = 42)}

\CommentTok{\#o retorno previsto dos ativos é a média de todos os cenários simulados}
\CommentTok{\#resultSimu \textless{}{-} simuModel |\textgreater{} fitted() |\textgreater{} as.data.frame() }
\CommentTok{\#resultSimu \textless{}{-} resultSimu |\textgreater{} mutate(rMedio = rowMeans(resultSimu)) |\textgreater{} rownames\_to\_column("dataFutura")}

\CommentTok{\#resultModel \textless{}{-} resultSimu |\textgreater{} select(dataFutura, rMedio)}

\CommentTok{\#saveResult \textless{}{-} saveResult |\textgreater{} bind\_rows(tibble(dataFutura=resultModel$dataFutura,Ativo=modeloFinal[i,]$Ativo, rPrevisto=resultModel$rMedio))}

\CommentTok{\#\}}
\end{Highlighting}
\end{Shaded}

\hypertarget{resultados}{%
\subsection{Resultados}\label{resultados}}

As predições são geradas em um arquivo .csv para posterior manipulação
via excel.

\begin{Shaded}
\begin{Highlighting}[]
\CommentTok{\#saveResult |\textgreater{} write.csv("resultfinal.csv") }
\end{Highlighting}
\end{Shaded}

Geramos os gráficos de predições (série a direita da linha vertical).

\begin{Shaded}
\begin{Highlighting}[]
\NormalTok{saveResult}\OtherTok{\textless{}{-}}\FunctionTok{read.csv}\NormalTok{(}\StringTok{"resultfinal.csv"}\NormalTok{)}
\NormalTok{saveResult}\OtherTok{\textless{}{-}}\NormalTok{ saveResult}\SpecialCharTok{|\textgreater{}} \FunctionTok{select}\NormalTok{(}\SpecialCharTok{{-}}\NormalTok{X)}
\NormalTok{dataAtual }\OtherTok{\textless{}{-}}\NormalTok{ bondsTibble}\SpecialCharTok{$}\NormalTok{Dates }\SpecialCharTok{|\textgreater{}} \FunctionTok{max}\NormalTok{()}
\NormalTok{anoAtual }\OtherTok{\textless{}{-}}\NormalTok{  bondsTibble}\SpecialCharTok{$}\NormalTok{Dates }\SpecialCharTok{|\textgreater{}} \FunctionTok{year}\NormalTok{() }\SpecialCharTok{|\textgreater{}} \FunctionTok{max}\NormalTok{()}

\NormalTok{resultGraph }\OtherTok{\textless{}{-}}\NormalTok{ saveResult }\SpecialCharTok{|\textgreater{}} \FunctionTok{group\_by}\NormalTok{(Ativo) }\SpecialCharTok{|\textgreater{}} \FunctionTok{mutate}\NormalTok{(}\AttributeTok{diaFuturo=}\FunctionTok{row\_number}\NormalTok{())}
\NormalTok{resultGraph }\OtherTok{\textless{}{-}}\NormalTok{ resultGraph }\SpecialCharTok{|\textgreater{}} \FunctionTok{mutate}\NormalTok{(}\AttributeTok{Dates=}\FunctionTok{add.bizdays}\NormalTok{(dataAtual,diaFuturo))}
\NormalTok{resultGraph }\OtherTok{\textless{}{-}}\NormalTok{ resultGraph }\SpecialCharTok{|\textgreater{}} \FunctionTok{rename}\NormalTok{(}\AttributeTok{rDiario=}\NormalTok{rPrevisto)}

\NormalTok{passado }\OtherTok{\textless{}{-}}\NormalTok{ bondsTibble }\SpecialCharTok{|\textgreater{}} \FunctionTok{select}\NormalTok{(Dates, Ativo, rDiario) }\SpecialCharTok{|\textgreater{}} \FunctionTok{group\_by}\NormalTok{(Ativo)}
\NormalTok{futuro }\OtherTok{\textless{}{-}}\NormalTok{resultGraph }\SpecialCharTok{|\textgreater{}} \FunctionTok{select}\NormalTok{(Dates, Ativo, rDiario)}
\NormalTok{consolidado}\OtherTok{\textless{}{-}}\FunctionTok{bind\_rows}\NormalTok{(passado,futuro)}

\NormalTok{consolidado }\SpecialCharTok{|\textgreater{}}\NormalTok{ dplyr}\SpecialCharTok{::}\FunctionTok{filter}\NormalTok{(}\FunctionTok{year}\NormalTok{(Dates)}\SpecialCharTok{\textgreater{}=}\NormalTok{ anoAtual}\DecValTok{{-}1}\NormalTok{) }\SpecialCharTok{|\textgreater{}} 
               \FunctionTok{ggplot}\NormalTok{(}\FunctionTok{aes}\NormalTok{(}\AttributeTok{x=}\NormalTok{Dates, }\AttributeTok{y=}\NormalTok{rDiario, }\AttributeTok{color=}\NormalTok{Ativo))}\SpecialCharTok{+}
               \FunctionTok{geom\_line}\NormalTok{()}\SpecialCharTok{+}
               \FunctionTok{geom\_vline}\NormalTok{(}\AttributeTok{xintercept=}\NormalTok{dataAtual)}\SpecialCharTok{+}
               \FunctionTok{facet\_wrap}\NormalTok{(}\SpecialCharTok{\textasciitilde{}}\NormalTok{Ativo,}\AttributeTok{ncol =} \DecValTok{2}\NormalTok{)}
\end{Highlighting}
\end{Shaded}

\includegraphics{ForecastBonds_files/figure-latex/result graphs-1.pdf}

\end{document}
